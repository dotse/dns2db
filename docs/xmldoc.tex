
\documentclass[a4paper]{article}

\usepackage{alltt}

\setlength\textwidth{6.0in}
\setlength\oddsidemargin{0.5in}
\setlength\evensidemargin{0.5in}
%\setlength\parindent{0.25in}
%\setlength\parskip{0.25in} 

\newcommand{\urlb}{\footnotesize\begin{alltt}} 
\newcommand{\urle}{\end{alltt}\normalsize} 
\newcommand{\exampleb}{\small\begin{alltt}} 
\newcommand{\examplee}{\end{alltt}\normalsize} 
\renewcommand{\familydefault}{\sfdefault}
\newcommand{\xmlv}[1]{\textbf{\textless#1\textgreater}} 
\newcommand{\xml}[1]{\textless#1\textgreater} 
\newcommand{\dnsdb}{DNS\textsuperscript{2db}} 



\title{\dnsdb \linebreak\linebreak HTTP interface specification\linebreak \textbf{Draft} }

\author{\linebreak.SE (The Internet Infrastructure Foundation)}
\date{December 2009}
\begin{document}
\maketitle
\newpage

\tableofcontents
\newpage

\section{Introduction}
This document describes the http get request and the resulting output of 
the two \dnsdb \hspace{3pt}php scripts. For a description of the complete \dnsdb \hspace{3pt}system refer to the \dnsdb 
whitepaper which can be found at opensource.iis.se.

The dns2db.php script is a proxy that connects to one or several collector 
nodes in the \dnsdb  system and aggregates their output. 
The dns2dbnode.php is the script that runs on the collector nodes and retrieves data 
from the \dnsdb databases has almost the same interface as dns2db.php. 
The main purpose of the dns2db.php script is to feed the flex gui with data.


\section{Overview}

This section contains a brief overview of the get parameters and the 
xml output tags.

\subsection{Query parameters}
\begin{center}
    \begin{tabular}{ | l | p{10cm} |}
    \hline
    \textbf{Parameter} & \textbf{Format}  \\ \hline
    
    function & 
    Specifies the main function of the script se a discussion of each function in the functions section. 
    \\ \hline

    nodes &  Comma separated list of servers to be applied to the query.
    
    Format: \verb|&nodes=node1,node2,node3,node4|
    where \verb|node1| should be the nodename as output by the nodelist function. 
    \\ \hline

    day & Specifies the date

    Format: \verb|&day=YYYYMMDD|
    \\ \hline

    time & Specifies the start time of the interval

    Format: \verb|&time=HHMM| 
    \\ \hline

    count & Specifies the maximum number of returned entries. This does not affect the serverlist function.

    Format: \verb|&count=n|
    \\ \hline

    resolver & 
    Specifies the resolver for the domainforresolver function.
    \linebreak Format for ipv6: \verb|&resolver=nnnn:nnnn::nnnn| 
    
    
    Format for ipv4: \verb|&resolver=::nnn.nnn.nnn.nnn|
    \\ \hline

    domain & 
    Specifies the domain for the resolverfordomain function

    Format: \verb|&domain=foo.se|
    \\ \hline
    \end{tabular}
\end{center}


\newpage

\subsection{The XML output}


The response of all querys is in xml format and contain only the data requested. The following tags are in use.


\begin{center}
    \begin{tabular}{ | l | p{10cm} |}
        \hline
        \textbf{Tag} & \textbf{Function}  \\ \hline
        \verb|<items>|&
        list container  \\ \hline
        \verb|<item>|&
        container for each list item \\ \hline
        \verb|<position>|&
        index of the list item \\ \hline
        \verb|<qcount>|&
        number of queries per minute\\ \hline
        \verb|<domain>|&
        domain name\\ \hline
        \verb|<displaytext>|&
        displaytext\\ \hline
        \verb|<status>|&
        contains status information for the nodes in the system \\ \hline
        \verb|<node>|&
        a status entry per node inside the status tag \\ \hline
        \verb|<filter>|&
        available filters, only used in the response to the filterlist function \\ \hline
        \verb|<server>|&
        node description, only used in the response to the nodelist function \\ \hline
    \end{tabular}
\end{center}


\newpage
\section{Functions}

    \subsection{Function: nodelist}

	The nodelist function returns an xml formatted list with available DNS2db collection nodes.\\
	The nodelist function is not available from the dns2dbnode.php script.

\begin{center}
    \begin{tabular}{ | l | p{6cm} |}
    \hline
    \textbf{Parameter} & \textbf{Comment}  
    \\ \hline
    function
    &
    must be "nodelist"
    \\ \hline
    \end{tabular}
\end{center}

Example:

\urlb
http://servername/dns2db.php?function=nodelist
\urle

Example output:
\exampleb
<?xml version="1.0" encoding="ISO-8859-1">
<items>
 <server name="a" dnsname="a.ns.tld" displayname="A" description="Big server"/>
 <server name="a" dnsname="b.ns.tld" displayname="B" description="Small server"/>
</items>
\examplee
     
     
\newpage
\subsection{Function: filterlist}

	The filterlist function returns an xml formatted list with available filters.
	A gui should ideally list the returned \xml{filter} tags as a set of check boxes and combobuttons. 		
	Any filter item with an opts attribute should be considered a combobutton or some kind of multiselect.
	The filters are not configurable but the number of filters and their names may change with future versions of \dnsdb.\\
	The filterlist function is not available from the dns2dbnode.php script.

\begin{center}
    \begin{tabular}{ | l | p{6cm}|}
    \hline
    \textbf{Parameter} & \textbf{Comment}  
    \\ \hline
    function
    &
    must be "filterlist"
    \\ \hline

    \end{tabular}
\end{center}

Example:

\urlb
http://server/dns2db.php?function=filterlist
\urle

Example output:

\exampleb
<?xml version="1.0" encoding="ISO-8859-1"?> 
<items> 
  <filter name="tcp" code="T" default="1"/> 
  <filter name="udp" code="U" default="1"/> 
  <filter name="v4" code="4" default="1"/> 
  <filter name="v6" code="6" default="1"/> 
  <filter name="qtype" code="QT" default="ALL"
      opts="ALL,A,NS,MD,MF,CNAME,SOA,MB,MG,MR,NULL,WKS,PTR,HINFO,MINFO,MX,TXT,
          RP,AFSDB,X25,ISDN,RT,NSAP,NSAP-PTR,SIG,KEY,PX,GPOS,AAAA,LOC,NXT,EID,
          NIMLOC,SRV,ATMA,NAPTR,KX,CERT,A6,DNAME,SINK,OPT,APL,DS,SSHFP,IPSECKEY,
          RRSIG,NSEC,DNSKEY,DHCID,SPF,UINFO,UID,GID,UNSPEC,TKEY,TSIG,IXFR,AXFR,
          MAILB,MAILA,*,TA,DLV,"/> 
</items> 
\examplee
\newpage
\subsection{Function: topresolvers}
	The topresolvers function returns an xml formatted list of the top resolvers.

\begin{center}
    \begin{tabular}{ | l | p{6cm}|}
    \hline
    \textbf{Parameter} & \textbf{Comment}  
    \\ \hline
    function
    &
    must be "topresolvers"
    \\ \hline
    day
    &
    Required
    \\ \hline
    time
    &
    Required
    \\ \hline
    count
    &
    Required
    \\ \hline
    nodes
    &
    Optional, not used by dns2dbnode.php
    \\ \hline
    filters
    &
    Optional, comma separated list of filters to be used.
    \\ \hline
    \end{tabular}
\end{center}

Example:

\urlb
http://server/dns2db.php?function=topresolvers&day=20081224&time=1500&count=2&nodes=a,b
\urle

Example output:
\exampleb
<?xml version="1.0" encoding="ISO-8859-1">
<items>
 <item>
  <position>1</position>
  <qcount>48</qcount>
  <domain>ip-address1</domain>
  <displaytext>hostname1</displaytext>
 </item>
 <item>
  <position>2</position>
  <qcount>6</qcount>
  <domain>ip-address2</domain>
  <displaytext>hostname2</displaytext>
 </item>
 <status> 
  <node name="a" result="1" /> 
  <node name="b" result="1" /> 
 </status> 
</items>
\examplee
\newpage
\subsection{Function: topdomains}
	The topdomains function returns an xml formatted list of the top domains.

\begin{center}
    \begin{tabular}{ | l | p{6cm}|}
    \hline
    \textbf{Parameter} & \textbf{Comment}  
    \\ \hline
    function
    &
    must be "topdomains"
    \\ \hline
    day
    &
    Required
    \\ \hline
    time
    &
    Required
    \\ \hline
    count
    &
    Required
    \\ \hline
    nodes
    &
    Optional, not used by dns2dbnode.php
    \\ \hline
    filters
    &
    Optional, comma separated list of filters to be used.
    \\ \hline
    \end{tabular}
\end{center}

Example:

\urlb
http://servername/dns2db.php?function=topdomains&day=20081224&time=1500&count=2&nodes=a,b
\urle

Example output:
\exampleb
<?xml version="1.0" encoding="ISO-8859-1">
<items>
 <item>
  <position>1</position>
  <qcount>48</qcount>
  <domain>foo.se.</domain>
  <displaytext>foo.se.</displaytext>
 </item>
 <item>
  <position>2</position>
  <qcount>6</qcount>
  <domain>ns.se.</domain>
  <displaytext>ns.se</displaytext>
 </item>
 <status> 
  <node name="a" result="1" /> 
  <node name="b" result="1" /> 
 </status> 
</items>
\examplee
\newpage
\subsection{Function: domainforresolver}
    
	The domainforresolver function returns an xml formatted list of the top domains for a specific resolver.

\begin{center}
    \begin{tabular}{ | l | p{6cm}|}
    \hline
    \textbf{Parameter} & \textbf{Comment}  
    \\ \hline
    function
    &
    must be "domainforresolver"
    \\ \hline
    resolver
    &
    Required
    \\ \hline    
    day
    &
    Required
    \\ \hline
    time
    &
    Required
    \\ \hline
    count
    &
    Required
    \\ \hline
    nodes
    &
    Optional, not used by dns2dbnode.php
    \\ \hline
    filters
    &
    Optional, comma separated list of filters to be used.
    \\ \hline
    \end{tabular}
\end{center}


Example:

\urlb
http://server/dns2db.php?function=domainforresolver&day=20081224&time=1500
                          &count=2&resolver=::f00d:1234&filters=T,U,4,6,QT:ALL
\urle

Example output:
\exampleb
<?xml version="1.0" encoding="ISO-8859-1">
<items>
 <item>
  <position>1</position>
  <qcount>100</qcount>
  <domain>foo.se.</domain>
  <displaytext>foo.se.</displaytext>
 </item>
 <item>
  <position>2</position>
  <qcount>67</qcount>
  <domain>bar.se.</domain>
  <displaytext>bar.se.</displaytext>
 </item>
 <status> 
  <node name="a" result="1" /> 
  <node name="b" result="1" /> 
 </status> 
</items>
\examplee
\newpage
        \subsection{Function: resolversfordomain}
	The resolversfordomain function returns an xml formatted list of the top resolvers that have asked for a specific domain.


\begin{center}
    \begin{tabular}{ | l | p{6cm}|}
    \hline
    \textbf{Parameter} & \textbf{Comment}  
    \\ \hline
    function
    &
    must be "resolversfordomain"
    \\ \hline
    domain
    &
    Required
    \\ \hline    
    day
    &
    Required
    \\ \hline
    time
    &
    Required
    \\ \hline
    count
    &
    Required
    \\ \hline
    nodes
    &
    Optional, not used by dns2dbnode.php
    \\ \hline
    filters
    &
    Optional, comma separated list of filters to be used.
    \\ \hline
    \end{tabular}
\end{center}

Example:
\urlb
http://servername/dns2db.php?function=resolversfordomain&day=20081224&time=1500&count=2&domain=foo.se
\urle

Example output:
\exampleb
<?xml version="1.0" encoding="ISO-8859-1">
<items>
 <item>
  <position>1</position>
  <qcount>4</qcount>
  <domain>ip-address1</domain>
  <displaytext>hostname1</displaytext>
 </item>
 <item>
  <position>2</position>
  <qcount>3</qcount>
  <domain>ip-address2</domain>
  <displaytext>hostname2</displaytext>
 </item>
 <status>
  <node name="a" result="1" /> 
  <node name="b" result="1" /> 
 </status> 
</items>
\examplee

\newpage




\section{XML Tags}

    \subsection{Tag:\xml{items}  }
    
    The \xmlv{items} tag is the XML root tag for all responses in \dnsdb 


    
    \begin{center}
        \begin{tabular}{ | p{2.1cm} | p{2cm} | p{9cm} |}
        \hline
        \textbf{Attribute} & \textbf{Importance} & \textbf{Comment}  
        \\ \hline
        none & &
        \\ \hline
        \end{tabular}
        
        \small Attributes
        
        \begin{tabular}{ | p{2.1cm} | p{2cm} | p{9cm} |}
        \hline
        \textbf{Subtags} & \textbf{Importance} & \textbf{Comment}  
        \\ \hline
        \xmlv{item}
        & 
        See comment
        &
        Required for all functions except nodelist or filterlist
        \\ \hline
        \xmlv{status}
        &
        Optional
        &
        for all functions except nodelist or filterlist
        \\ \hline
        \xmlv{filter}
        &
        See comment
        &
        Required only for the filterlist function
        \\ \hline
        \xmlv{server}
        &
        See comment
        &
        Required only for the nodelist function
        \\ \hline
        \end{tabular}
        
        \small Subtags
    \end{center}    


    
    \subsection{Tag:\xml{item}  }
    
    The item tag is a subtag of \xmlv{items} that contains the 
    resulting rows from the database query. Found in the \xml{items} root tag.
    
    \begin{center}
        \begin{tabular}{ | p{2.1cm} | p{2cm} | p{9cm} |}
        \hline
        \textbf{Attribute} & \textbf{Importance} & \textbf{Comment}  
        \\ \hline
        none & &
        \\ \hline
        \end{tabular}
        
        \small Attributes
        
        \begin{tabular}{ | p{2.1cm} | p{2cm} | p{9cm} |}
        \hline
        \textbf{Subtags} & \textbf{Importance} & \textbf{Comment}  
        \\ \hline
        \xmlv{position}
        &
        Required
        &
        \\ \hline
        \xmlv{qcount}
        &
        Required
        &
        \\ \hline
        \xmlv{domain}
        &
        Required
        &
        
        \\ \hline
        \xmlv{displaytext}
        &
        Required
        &
        
        \\ \hline
        \end{tabular}
        
        \small Subtags
    \end{center}        
    
    \subsection{Tag:\xml{position}  }
        Contains a simple counter specifiying the order of the rows. Always starts at 1 and increases for each item.\\ 
        Specified as tag content only and found in the \xml{item} tag.

    \begin{center}
        \begin{tabular}{ | p{2.1cm} | p{2cm} | p{9cm} |}
        \hline
        \textbf{Attribute} & \textbf{Importance} & \textbf{Comment}  
        \\ \hline
        none & &
        \\ \hline
        \end{tabular}
        
        \small Attributes
        
        \begin{tabular}{ | p{2.1cm} | p{2cm} | p{9cm} |}
        \hline
        \textbf{Subtags} & \textbf{Importance} & \textbf{Comment}  
        \\ \hline
        none
        &
        &
        \\ \hline
        \end{tabular}
        
        \small Subtags
    \end{center} 
    
    
    \subsection{Tag:\xml{qcount}  }
        Specifies the number of occurrances of this item. i.e. number of queries for a specific domain.\\ 
        Specified as tag content only and found in the \xml{item} tag.

    \begin{center}
        \begin{tabular}{ | p{2.1cm} | p{2cm} | p{9cm} |}
        \hline
        \textbf{Attribute} & \textbf{Importance} & \textbf{Comment}  
        \\ \hline
        none & &
        \\ \hline
        \end{tabular}
        
        \small Attributes
        
        \begin{tabular}{ | p{2.1cm} | p{2cm} | p{9cm} |}
        \hline
        \textbf{Subtags} & \textbf{Importance} & \textbf{Comment}  
        \\ \hline
        none
        &
        &
        \\ \hline
        \end{tabular}
        
        \small Subtags
    \end{center} 

    \subsection{Tag:\xml{domain}  }
        Specifies a domain or somewhat misleading host depending on the current function.\\ 
        Specified as tag content only and found in the \xml{item} tag.

    \begin{center}
        \begin{tabular}{ | p{2.1cm} | p{2cm} | p{9cm} |}
        \hline
        \textbf{Attribute} & \textbf{Importance} & \textbf{Comment}  
        \\ \hline
        none & &
        \\ \hline
        \end{tabular}
        
        \small Attributes
        
        \begin{tabular}{ | p{2.1cm} | p{2cm} | p{9cm} |}
        \hline
        \textbf{Subtags} & \textbf{Importance} & \textbf{Comment}  
        \\ \hline
        none
        &
        &
        \\ \hline
        \end{tabular}
        
        \small Subtags
    \end{center} 
  
    \subsection{Tag:\xml{displaytext}  }
        The content of \xml{displaytext} is either the same as found in 
        \xml{domain} or a humanreadable version of whats found in \xml{domain} 
        such as the hostname for an ip adress.\\ 
        Specified as tag content only and found in the \xml{item} tag.

    \begin{center}
        \begin{tabular}{ | p{2.1cm} | p{2cm} | p{9cm} |}
        \hline
        \textbf{Attribute} & \textbf{Importance} & \textbf{Comment}  
        \\ \hline
        none & &
        \\ \hline
        \end{tabular}
        
        \small Attributes
        
        \begin{tabular}{ | p{2.1cm} | p{2cm} | p{9cm} |}
        \hline
        \textbf{Subtags} & \textbf{Importance} & \textbf{Comment}  
        \\ \hline
        none
        &
        &
        \\ \hline
        \end{tabular}
        
        \small Subtags
    \end{center} 

    \subsection{Tag:\xml{status}  }
        The \xml{status} tag is included under the \xml{items} tags after the 
        \xml{item} tags and is used to specify which nodes has answered the query.

    \begin{center}
        \begin{tabular}{ | p{2.1cm} | p{2cm} | p{9cm} |}
        \hline
        \textbf{Attribute} & \textbf{Importance} & \textbf{Comment}  
        \\ \hline
        none & &
        \\ \hline
        \end{tabular}
        
        \small Attributes
        
        \begin{tabular}{ | p{2.1cm} | p{2cm} | p{9cm} |}
        \hline
        \textbf{Subtags} & \textbf{Importance} & \textbf{Comment}  
        \\ \hline
        \xmlv{node}
        &
        & A node tag would be expected here but if it's not there it should not be treated as an error.
        \\ \hline
        \end{tabular}
        
        \small Subtags
    \end{center}        



    \subsection{Tag:\xml{node}  }
    The \xml{node} tag is used to specify which nodes has responded to a query. 
    This is currently used by the \dnsdb flash interface to tag nodes as green 
    or red depending on wheter they returned data or not. 
    Found in the \xml{status} tag.
    \begin{center}
        \begin{tabular}{ | p{2.1cm} | p{2cm} | p{9cm} |}
        \hline
        \textbf{Attribute} & \textbf{Importance} & \textbf{Comment}  
        \\ \hline
        name
        &
        Required
        &
        Specifies the name of a node
        \\ \hline
        result
        &
        Required
        &
        result is 1 when the node has returned data otherwise 0 
        \\ \hline
        \end{tabular}
        
        \small Attributes
        
        \begin{tabular}{ | p{2.1cm} | p{2cm} | p{9cm} |}
        \hline
        \textbf{Subtags} & \textbf{Importance} & \textbf{Comment}  
        \\ \hline
        none
        & &
        \\ \hline
        \end{tabular}
        
        \small Subtags
    \end{center}        



    \subsection{Tag:\xml{filter}  }
        Speficies an available filter. Returned by the filterlist function 
        inside the \xml{items} tag.
    \begin{center}
        \begin{tabular}{ | p{2.1cm} | p{2cm} | p{9cm} |}
        \hline
        \textbf{Attribute} & \textbf{Importance} & \textbf{Comment}  
        \\ \hline
        name & Required & Specifies the name of the filter
        \\ \hline
        code & Required & Specifies the code to be used in the filter parameter string 
        \\ \hline
        default & Required & default value can be 0 or 1 or one of the values in opts
        \\ \hline
        opts & Optional & A string with comma separated values for the dropdown menu
        \\ \hline
        \end{tabular}
        \small Attributes
        
        \begin{tabular}{ | p{2.1cm} | p{2cm} | p{9cm} |}
        \hline
        \textbf{Subtags} & \textbf{Importance} & \textbf{Comment}  
        \\ \hline
        none
        & &
        \\ \hline
        \end{tabular}
        
        \small Subtags
    \end{center}        
        
        
    \subsection{Tag:\xml{server}  }
    
    Contains rows of output from running the nodelist function inside the 
    \xml{items} tag.

    \begin{center}
        \begin{tabular}{ | p{2.1cm} | p{2cm} | p{9cm} |}
        \hline
        \textbf{Attribute} & \textbf{Importance} & \textbf{Comment}  
        \\ \hline
        name & Required & The uniqe name of the node. Should contain characters and digits only.
        \\ \hline
        dnsname & Required & The fully qualitfied domain name of the node.
        \\ \hline
        displayname & Required & The name to display.
        \\ \hline
        description & Required & A string describing the node (may be empty).
        \\ \hline
        \end{tabular}
        \small Attributes
        
        \begin{tabular}{ | p{2.1cm} | p{2cm} | p{9cm} |}
        \hline
        \textbf{Subtags} & \textbf{Importance} & \textbf{Comment}  
        \\ \hline
        none
        & &
        \\ \hline
        \end{tabular}
        
        \small Subtags
    \end{center}        


\end{document}

